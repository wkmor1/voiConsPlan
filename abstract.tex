\clearpage
\subsection*{Abstract}\label{abstract}
\addcontentsline{toc}{subsection}{Abstract}

Spatial conservation plans are typically based on uncertain inputs and additional data may improve them. Value of information analyses indicate how beneficial new information is to delivering a better outcome in a decision problem. However, the toolset of spatial conservation prioritization does not yet contain a method for assessing the value of new information to a spatial conservation plan. If the value of information could be calculated, then we would know the benefit from the optimal allocation of effort to gather new information. Here, for the first time we demonstrate how a formal value of information analysis can be applied to a spatial conservation plan for a real-world setting. We show how a value of information analysis can be combined with traditional conservation planning tools to determine the benefit of new information on species distributions and optimize a reserve network to protect them. We incorporate uncertainty into conservation planning with Monte Carlo sampling of the planning inputs and then test the effects of uncertainty reduction to calculate the value of additional information to a conservation plan. Optimally incorporating new information into conservation plans will reduce the loss of resources spent on unnecessary information gathering where new data has no or little benefit to the fundamental objectives of the plan. In our case study, we found that reducing uncertainty in threatened species distributions, by including new information, will increase the expected performance of a conservation plan allowing for a greater level of protection while maintaining the same cost of protecting habitat. There are two clear benefits of combining value of information analysis with spatial conservation planning. One is that it allows the planner to identify three categories of sites: those that warrant investment by acquisition (or other conservation action) but not further information gathering, those that may warrant investment after further investigation and those that warrant neither investment in any conservation action or in gathering more information. Second, in considering uncertainty about biodiversity features formally, VOI analyses help planners to evaluate the importance of their uncertainty in the currency of the performance measure of their plan's objective.
